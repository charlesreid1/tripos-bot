The problem of finding the accurate value of~$\omega_{n}$ in a numerical form
involving square roots only, as in the formula $\omega_{3} = \frac{1}{2}(-1 + i\sqrt{3})$, is the
algebraical equivalent of the geometrical problem of inscribing a regular
polygon of $n$~sides in a circle of unit radius by Euclidean methods, \ie\ by ruler
and compasses. For this construction will be possible if and only if we can
construct lengths measured by $\cos(2\pi/n)$ and $\sin(2\pi/n)$; and this is possible
(\Ref{Ch.}{II}, \MiscExs{II}~22) if and only if these numbers are expressible in a form
involving square roots only.

Euclid gives constructions for $n = 3$, $4$, $5$, $6$, $8$, $10$, $12$, and~$15$. It is
evident that the construction is possible for any value of~$n$ which can be
found from these by multiplication by any power of~$2$. There are other
special values of~$n$ for which such constructions are possible, the most interesting
being~$n = 17$.
