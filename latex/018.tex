The coefficients of the transformation $z = (aZ + b)/(cZ + d)$ are subject
to the condition $ad - bc = 1$.    Show that, if $c \neq 0$, there are two \emph{fixed points}
$\alpha$,~$\beta$, i.e., points unaltered by the transformation, except when $(a + d)^{2} = 4$, when
there is only one fixed point~$\alpha$; and that in these two cases the transformation
may be expressed in the forms
\[
\frac{z - \alpha}{z - \beta} = K\frac{Z - \alpha}{Z - \beta},\quad
\frac{1}{z - \alpha} = \frac{1}{Z - \alpha} + K.
\]

Show further that, if $c = 0$, there will be one fixed point~$\alpha$ unless $a = d$,
and that in these two cases the transformation may be expressed in the
forms
\[
z - \alpha = K(Z - \alpha),\quad
z = Z + K.
\]

Finally, if $a$,~$b$,~$c$,~$d$ are further restricted to positive integral values (including
zero), show that the only transformations with less than two fixed
points are of the forms $(1/z) = (1/Z) + K$, $z = Z + K$. \MathTrip{1911.}

